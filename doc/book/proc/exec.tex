\chapter{Executable File Formats}
\label{chap:exec}

\section{Introduction to executables in Zeke}

The kernel has a support for implementing loader functions for any new
executable format but currently only 32bit \acf{elf} support exist.
Loading of a new process image is invoked by calling exec syscall call that
calls \verb+load_image()+ function in \verb+kern/exec.c+ file.
Process image loaders are advertised by using \verb+EXEC_LOADER+ macro.

\section{ELF32}

The \acs{elf} loader in Zeke can be used to load statically linked executables
as well as anything linked dynamically. Currently only two loadable sections can
be loaded, code region and heap region.

\subsection{Suported \acs{elf} sections}

The kernel reads process memory regions based on information provided by
\verb+PT_LOAD+ sections in the \acs{elf} file.

The kernel can read additional information needed for executing a binary
from the elf notes. The non-standard notes are only parsed if \verb+Zeke+
is defined as an owner of the note.

\verb+NT_VERSION+

\verb+NT_STACKSIZE+

\verb+NT_STACKIZE+ note can be used to hint the kernel about the preferred
minimum size for the main thread stack.

\verb+NT_CAPABILITIES+

\verb+NT_CAPABILITIES+ note is used to inform the kernel about capabilities
required to execute a binary file. The elf loader attempts to set each
capability as an effective capability, which may fail if the capability
isn't in the bounding capabilities set. In case the file has
\ver+O_EXEC_ALTPCAP+ oflag set then the loader will first add the capabilities
found in these notes to the bounding capabilities set, i.e. the executable
can gain the bounding capabilities.

\verb+NT_CAPABILITIES_REQ+

\verb+NT_CAPABILITIES__REQ+ note functions similar to  \verb+NT_CAPABILITIES+
but it doesn't allow new bounding capabilities to be gained even when the
binary file is opened with \ver+O_EXEC_ALTPCAP+.

The user space implementation of these note types is discussed in section
\ref{sec:libc_elf}.
