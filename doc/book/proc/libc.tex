\chapter{Libc: Process and Thread Management}

\section{\acs{elf} support}\label{sec:libc_elf}

Zeke libc supports adding simple note sections to \acs{elf} files by using
macros provided by \verb+sys/elf_notes.h+ header file. As discussed
previously in chapter \ref{chap:exec}, some additional information about the
runtime environment requirements can be passed via note sections. For example,
a macro called \verb+ELFNOTE_STACKSIZE+ can be used for indicating the minimum
stack size needed by the main tread of an executable, see listing
\ref{list:hugestack}.

Another important Zeke specific note type is \verb+NT_CAPABILITIES+ that
can be used to annotate the capabilities required to succesfully execute
a given elf binary file. The capability notes can be created using the
\verb+ELFNOTE_CAPABILITIES(...)+ macro. Each note can have 64
capabilities and the total number of these notes is unlimited.

\lstinputlisting[label=list:hugestack,%
caption=hugestack.c]{../../usr/examples/hugestack.c}

\section{Fork and exec}

\subsection{Creating a daemon}

Creating a daemon is probably one of the main features of operating systems
like Zeke. The preferred way of creating a daemon in Zeke is forking once and
creating a new session for the child process by calling \verb+setsid()+. The
parent process may exit immediately after forking. Listing \ref{list:daemon}
shows an example of a daemon creation procedure.

\lstinputlisting[label=list:daemon,%
caption=daemon.c]{../../usr/examples/daemon.c}

\section{User credentials control}

\section{Pthread}

\subsection{Thread creation and destruction}

\subsection{Thread local storage}

\subsection{Mutex}

\epigraph{Tis in my memory lock'd,\newline
          And you yourself shall keep the key of it.}{\textit{Hamlet}, 1, ii}

