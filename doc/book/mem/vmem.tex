\chapter{Virtual Memory}

Each process owns their own master page table, in contrast to some kernels where
there might be only one master page table or one partial master page table,
and varying number of level two page tables. The kernel, also known as proc 0,
has its own master page table that is used when a process executes in kernel
mode, as well as when ever a kernel thread is executing. Static or fixed page
table entries are copied to all master page tables created. A process shares its
master page table with its childs on \verb+fork()+ while \verb+exec()+ will
trigger a creation of a new master page table.

Virtual memory is managed as virtual memory buffers (\verb+struct buf+) that
are suitable for in-kernel buffers, IO buffers as well as user space memory
mappings. Additionlly the buffer system supports copy-on-write as well as
allocator schemes where a part of the memory is stored on a secondary
storage (i.e. paging).

Due to the fact that \verb+buf+ structures are used in different allocators
there is no global knowledge of the actual state of a particular allocation,
instead each allocator should/may keep track of allocation structs if desired
so. Ideally the same struct can be reused when moving data from a secondary
storage allocator to vralloc memory (physical memory). However we
always know whether a buffer is currently in core or not (\verb+b_data+) and
we also know if a buffer can be swapped to a different allocator
(\verb+B_BUSY+ flag).

\section{Page Fault handling and VM Region virtualization}

\begin{enumerate}
\item DAB exception transfers execution to \verb+interrupt_dabt+ in \verb+XXX_int_handlers.S+
\item \verb+mmu_data_abort_handler()+ (\verb+XXX_mmu.c+) gets called
\item to be completed...
\end{enumerate}
