\part{System Calls}

\chapter{Introduction to System Calls}

\section{Syscall flow sequence}

\begin{enumerate}
\item User scope thread makes a syscall by calling:
      \verb+syscall(SYSCALL_XXX_YYY, &args)+, where XXX is generally a
      module/compilation unit name, YYY is a function name and args is a
      syscall dataset structure in format declared in \verb+syscalldef.h+.

\item The interrupt handler calls \verb+_intSyscall_handler()+ function where
      syscall handler of the correct subsystem is resolved from
      \verb+syscall_callmap+.

\item Execution enters to the subsystem specific \verb+XXX_syscall()+
      function where the system call is either handled directly or a next level
      system call function is called according to minor number of
      the system call.

\item \verb+XXX_syscall()+ returns a \verb+uint32_t+ value which is, after
      multiple return steps, returned back to the caller which should know
      what type the returned value actually represents. In the future return
      value should be always integer value having the same size as register
      in the architecture. Everything else shall be passed both ways by using
      args structs, thus unifying the return value.
\end{enumerate}


\section{Syscall Major and Minor codes}

System calls are divided to major and minor codes so that major codes represents
a set of functions related to each other, usually all the syscall functions in a
single compilation unit. Major number sets are internally called groups. Both
numbers are defined in \verb+syscall.h+ file.


\chapter{Adding new syscalls and handlers}

\section{A new syscall}

\begin{itemize}
\item \verb+include/syscall.h+ contains syscall number definitions
\item \verb+include/syscalldef.h+ contains some useful structures that can be used when
      creating a new syscall
\item Add the new syscall under a syscall group handler
\end{itemize}

\section{A new syscall handler}

\begin{itemize}
\item Create a new syscall group into \verb+include/syscall.h+
\item Create syscall number definitions into the previous file
\item Add the new syscall group to the list of syscall groups in \verb+syscall.c+
\item Create a new syscall group handler
\end{itemize}


\chapter{sysctl}

The Zeke sysctl mechanism uses hierarchically organized \ac{MIB} tree as a
debugging and online configuration interface to the kernel. This is extremely
useful for example when testing scheduling parameters. Instead of recompiling
after every parameter change it is possible to change kernel's internal
parameters at run time by using sysctl interface.

There is only one syscall for sysctl which handles both reading/writing a
\ac{MIB} variable and queries to the MIB.

\section{Magic names}

There is some magic OID's that begins with \verb+{0,...}+ that are used for
queries and other special purposes. Particularly all OID's begin with 0 are
magic names. Currently allocated magic names are described in table
\ref{table:sysctlmagic}.

\begin{table}
\caption{sysctl magic names.}
\label{table:sysctlmagic}
\begin{tabular}{lll}
Name                & Internal function        & Purpose\\
\hline
\verb+{0,1,<iname>}+ & \verb+sysctl_sysctl_name()+     & Get the name of a MIB variable.\\
\verb+{0,2,<iname>}+ & \verb+sysctl_sysctl_next()+     & Get the next variable from MIB tree.\\
\verb+{0,3}+            & \verb+sysctl_sysctl_name2oid()+ & String name to integer name of the variable.\\
\verb+{0,4,<iname>}+ & \verb+sysctl_sysctl_oidfmt()+   & Get format and type of a MIB variable.\\
\verb+{0,5,<iname>}+ & \verb+sysctl_sysctl_oiddescr()+ & Get description string of a MIB variable.
\end{tabular}
\end{table}

\section{Adding new sysctl entries}

\subparagraph{Nodes}
New nodes containing sub-entries can be created with \verb+SYSCTL_NODE+ macro
like shown in listing \ref{list:sysctl_node}.

\lstinputlisting[label=list:sysctl_node,caption=Sysctl node macro.]{sys/sysctl_node.c}

In order to populate variables and nodes under newly created node the node
should be declared with \verb+SYSCTL_DECL(<name>);+, this can be done either in
\verb+sysctl.h+, in the header file of the subsystem/mode or locally in the
source code file. In the latter case the new node won't be available in
global scope.

\subparagraph{Variables}
\begin{itemize}
\item \verb+SYSCTL_STRING+
\item \verb+SYSCTL_INT+
\item \verb+SYSCTL_UINT+
\item \verb+SYSCTL_LONG+
\item \verb+SYSCTL_ULONG+
\item \verb+SYSCTL_COUNTER_U64+
\end{itemize}

\subparagraph{Procedures}
\lstinputlisting[label=list:sysctl_proc,caption=Adding a sysctl prodedure.]{sys/sysctl_proc.c}

\subparagraph{Feature test variables}
\verb+FEATURE(name, desc)+
