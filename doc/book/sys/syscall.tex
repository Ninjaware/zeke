\part{System Calls}

\chapter{Introduction to System Calls}

System calls are the main method of communication between the kernel and user
space applications. The word syscall is usually used as a shorthand for system
call in the context of Zeke and this book.

\section{System call flow}

\begin{enumerate}
\item User scope thread makes a syscall by calling:
      \verb+syscall(SYSCALL_XXX_YYY, &args)+, where XXX is generally a
      module/compilation unit name, YYY is a function name and args is a
      syscall dataset structure in format declared in \verb+syscalldef.h+.

\item The interrupt handler calls \verb+_intSyscall_handler()+ function where
      syscall handler of the correct subsystem is resolved from
      \verb+syscall_callmap+.

\item Execution enters to the subsystem specific \verb+XXX_syscall()+
      function where the system call is either handled directly or a next level
      system call function is called according to minor number of
      the system call.

\item \verb+XXX_syscall()+ returns a \verb+uint32_t+ value which is, after
      multiple return steps, returned back to the caller which should know
      what type the returned value actually represents. In the future return
      value should be always integer value having the same size as register
      in the architecture. Everything else shall be passed both ways by using
      args structs, thus unifying the return value.
\end{enumerate}

\section{Syscall Major and Minor codes}

System calls are divided to major and minor codes so that major codes represents
a set of functions related to each other, usually all the syscall functions in a
single compilation unit. Major number sets are internally called groups. Both
numbers are defined in \verb+syscall.h+ file.


\chapter{Adding new syscalls and handlers}

\section{A new syscall}

\begin{itemize}
\item \verb+include/syscall.h+ contains syscall number definitions
\item \verb+include/syscalldef.h+ contains some useful structures that can be used when
      creating a new syscall
\item Add the new syscall under a syscall group handler
\end{itemize}

\section{A new syscall handler}

\begin{itemize}
\item Create a new syscall group into \verb+include/syscall.h+
\item Create syscall number definitions into the previous file
\item Add the new syscall group to the list of syscall groups in \verb+syscall.c+
\item Create a new syscall group handler
\end{itemize}


\chapter{Libc: Process and Thread Management}

\section{\acs{elf} support}\label{sec:libc_elf}

Zeke libc supports adding simple note sections to \acs{elf} files by using
macros provided by \verb+sys/elf_notes.h+ header file. As discussed
previously in chapter \ref{chap:exec}, some additional information about the
runtime environment requirements can be passed via note sections. For example,
a macro called \verb+ELFNOTE_STACKSIZE+ can be used for indicating the minimum
stack size needed by the main tread of an executable, see listing
\ref{list:hugestack}.

\lstinputlisting[label=list:hugestack,%
caption=hugestack.c]{../../usr/examples/hugestack.c}

\section{Pthread}
\subsection{Mutex}

\epigraph{Tis in my memory lock'd,\newline
          And you yourself shall keep the key of it.}{\textit{Hamlet}, 1, ii}


