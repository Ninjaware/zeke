\chapter{\acf{rcu}}

\acf{rcu} is a low overhead synchronization method mainly used for
synchronization between multiple readers and rarely occuring writers.
\acs{rcu} can be considered to be a lightweight versioning system for
data structures that allows readers to access the old version of the
data as long as they hold a reference to the old data, while new
readers will get a pointer to the new version.

The \acs{rcu} implementation supports only single writer and multiple
readers by itself but the user may use other synchronization methods to
allow multiple writers. Unlike traditional implementations the \acs{rcu}
implentation in Zeke allows interrupts and sleeping on both sides.
Instead of relying on time-based grace periods the \acs{rcu} state is
changed only when necessary due to a write synchronization. The
implementation also supports timer based reclamation of resources similar
to the traditional callback API. In practice the algorithm is based on
atomically accessed clock variable and two counters.

Before going any further with describing the implementation, let's define the
terminology and function names used in this chapter

\begin{itemize}
  \item \verb+gptr+ is the global pointer referenced by both, readers and writer(s),
  \item \verb+rcu_assing_pointer()+ writes a new value to a \verb+gptr+,
  \item \verb+rcu_dereference_pointer()+ dereferences the value of a \verb+gptr+,
  \item \verb+rcu_read_lock()+ increments the number of readers counter,
  \item \verb+rcu_read_unlock()+ decrements the number of readers counter,
  \item \verb+rcu_call()+ register a callback to be called when all readers of the
        old version of a \verb+gptr+ are ready, and
  \item \verb+rcu_synchronize()+ block until all current readers are ready.
\end{itemize}

The global control variables for \acs{rcu} are defined as follows
\begin{itemize}
  \item \verb+clk+ is a one bit clock selecting the \acs{rcu} state,
  \item \verb+ctr0+ is a counter for readers accepted in the first state,
  \item \verb+ctr1+ is a counter for readers accepted in the second state, and
  \item \verb+rcu_reclaim_list[2]+ contains a list of callbacks per state to
        old the versions of the resources.
\end{itemize}

In the following description we assume that the \acs{rcu} is in the same
state as initially but it has already ran an undefined number of iterations,
therefore $\mathtt{clk} = 0$.

Both states work equally and the clock just swaps the functionality. In the
initial state \verb+ctr0+ is incremented on every call to
\verb+rcu_read_lock()+. Whereas \verb+rcu_read_unlock()+ always decrements
the same counter that the previous call to \verb+rcu_read_lock()+ incremented,
this is ensured by passing a state variable to the caller when
\verb+rcu_read_lock()+ returns. After acquiring the lock a reader may call
\verb+rcu_dereference_pointer()+ to get the value pointed by a \verb+gptr+.

The writer may call \verb+rcu_assing_pointer()+ to assing a new value to the
\verb+gptr+. After the pointer has been assigned the writer should call either
\verb+rcu_call()+ or\\ \verb+rcu_synchronize()+ to ensure there is no more
readers for the old data, and to free the old data. In general the latter is
preferred if the writer is allowed to block.

\verb+rcu_synchronize()+ is the only mechanism that will advance the global
\acs{rcu} clock variable. The function works in two stages and access to the
function is syncronized by a spinlock. In the first stage it waits until
there is no more readers on the second counter (\verb+ctr1+), and changes the
clock state. The second stage waits until there is no more readers on the
counter pointed by the previous clock state, in the initial case this means
\verb+ctr0+. When \verb+ctr0+ has reached zero the reclamation of resources
registered to the reclaim list of the first state can be executed. Note that
these resources are not the ones that were assigned to \verb+gptr+ pointers
when the clock was in zero state but the resources that were alloced when the
clock was previously in the current state set in the previous stage of
\verb+rcu_synchronize()+.
