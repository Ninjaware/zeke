\part{Bootstrapping}
\chapter{Bootloader}
\chapter{Kernel Initialization}

Kernel initialization order is defined as follows:
\begin{itemize}
    \item \verb+hw_preinit+
    \item \verb+kmalloc_init()+
    \item constructors
    \item \verb+hw_postinit+
    \item \verb+kinit()+
\end{itemize}

After kinit, scheduler will kick in and initialization continues in
\verb+sinit(8)+ process.

Every initializer function should contain \verb+SUBSYS_INIT("XXXX init");+ as
the first line of the function and optionally single or multiple
\verb+SUBSYS_DEP(YYYY_init);+ lines declaring subsystem initialization
dependencies.

\section{Kernel module/subsystem initializers}

There are four kind of initializers supported at the moment:

\begin{itemize}
    \item \textbf{hw\_preinit} for mainly hardware related initializers
    \item \textbf{hw\_postinit} for hardware related initializers
    \item \textbf{constructor} (or init) for generic initialization
\end{itemize}

descturctors are not yet supported in Zeke but if there ever will be LKM
support the destructors will be called when unloading the module.

Listing \ref{list:kmodinit} shows the traditional constructor/intializer
notation supported by Zeke.  

\lstinputlisting[label=list:kmodinit,%
caption=kmod.c]{boot/kmod.c}

% TODO The new way of init

Constructor prioritizing is not supported and \verb+SUBSYS_DEP+ should be used
instead to indicate initialization dependecies.

%\subsection{hw\_preinit and hw\_postinit}

hw\_preinit and hw\_postinit can be used by including \verb+kinit.h+ header file
and using the notation as shown in \ref{list:hwprepostinit}.

\lstinputlisting[label=list:hwprepostinit,%
caption=hwprepostinit.c]{boot/hwprepostinit.c}
